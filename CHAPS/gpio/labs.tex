\clearpage\section{Labs}\begin{Lab}


\begin{exe} {List all the PocketBeagle pins and their configuration}
   We'll use \verb?show-pins.pl? to do so.

   \begin{sol}
      \begin{raw}
perl /opt/scripts/device/bone/show-pins.pl -v
    
l \end{raw}
   \end{sol}
\end{exe}

\begin{exe} {Playing with the button}
   First we'll use the button to read the value of a GPIO input.

   The button on the Bacon Bits is on the 13th gpio of the second gpio chip
   (gpiochip1), although sysfs lists this as gpiochip32, since there are
   32 gpios per chip.

   So the gpio number is 32 + 13, or 45. gpio45 is on by default, but
   otherwise we'd have to first export it to make it available in syfs.

   Read both the settings and the value of gpio45 with the USR button both
   pressed and not pressed.

   \begin{sol}
      \begin{raw}
root@beaglebone:~# ls /sys/class/gpio/gpio45
active_low  device  direction  edge  label  power  subsystem  uevent  value
root@beaglebone:~# cat /sys/class/gpio/gpio45/active_low 
0
root@beaglebone:~# cat /sys/class/gpio/gpio45/direction 
in
root@beaglebone:~# cat /sys/class/gpio/gpio45/label 
P2_33
root@beaglebone:~# cat /sys/class/gpio/gpio45/value 
1
watch -n0 cat /sys/class/gpio/gpio45/value
      \end{raw}
   \end{sol}
\end{exe}

\begin{exe} {Read the button with libgpiod}
   Now use gpioget from libgpiod to read the button.

   \begin{sol}
      \begin{raw}
root@beaglebone:~# gpioget gpiochip1 13
1
      \end{raw}
   \end{sol}
\end{exe}

\begin{exe} {Light up the USR3 LED with the LED class}
   Now use /sys/class/led/* to light up the LED.

   \begin{sol}
      \begin{raw}
root@beaglebone:~# ls /sys/class/leds/beaglebone\:green\:usr3
brightness  device  max_brightness  power  subsystem  trigger  uevent
root@beaglebone:~# cat /sys/class/leds/beaglebone\:green\:usr3/brightness 
0
root@beaglebone:~# echo 255 > /sys/class/leds/beaglebone\:green\:usr3/brightness  
root@beaglebone:~# cat /sys/class/leds/beaglebone\:green\:usr3/brightness 
255
root@beaglebone:~# echo 0 > /sys/class/leds/beaglebone\:green\:usr3/brightness  
      \end{raw}
   \end{sol}
\end{exe}

\begin{exe} {Light up the USR3 LED with a LED trigger}

   \begin{sol}
      \begin{raw}
root@beaglebone:~# cat /sys/class/leds/beaglebone\:green\:usr3/trigger 
[none] rc-feedback rfkill-any kbd-scrolllock kbd-numlock kbd-capslock
kbd-kanalock kbd-shiftlock kbd-altgrlock kbd-ctrllock kbd-altlock
kbd-shiftllock kbd-shiftrlock kbd-ctrlllock kbd-ctrlrlock usb-gadget
usb-host mmc0 timer oneshot disk-activity ide-disk mtd nand-disk
heartbeat backlight gpio cpu cpu0 default-on panic 
root@beaglebone:~# echo heartbeat > /sys/class/leds/beaglebone\:green\:usr3/trigger
root@beaglebone:~# cat /sys/class/leds/beaglebone\:green\:usr3/trigger 
none rc-feedback rfkill-any kbd-scrolllock kbd-numlock kbd-capslock
kbd-kanalock kbd-shiftlock kbd-altgrlock kbd-ctrllock kbd-altlock
kbd-shiftllock kbd-shiftrlock kbd-ctrlllock kbd-ctrlrlock usb-gadget
usb-host mmc0 timer oneshot disk-activity ide-disk mtd nand-disk
[heartbeat] backlight gpio cpu cpu0 default-on panic 
root@beaglebone:~# echo none > /sys/class/leds/beaglebone\:green\:usr3/trigger
      \end{raw}
   \end{sol}
\end{exe}

\begin{exe} {Light up the USR3 LED with the button as a LED trigger}

   \begin{sol}
      \begin{raw}
root@beaglebone:~# cat /sys/class/leds/beaglebone\:green\:usr3/trigger 
[none] rc-feedback rfkill-any kbd-scrolllock kbd-numlock kbd-capslock
kbd-kanalock kbd-shiftlock kbd-altgrlock kbd-ctrllock kbd-altlock
kbd-shiftllock kbd-shiftrlock kbd-ctrlllock kbd-ctrlrlock usb-gadget
usb-host mmc0 timer oneshot disk-activity ide-disk mtd nand-disk
heartbeat backlight gpio cpu cpu0 default-on panic 
root@beaglebone:~# echo gpio > /sys/class/leds/beaglebone\:green\:usr3/trigger
root@beaglebone:~# cat /sys/class/leds/beaglebone\:green\:usr3/trigger 
none rc-feedback rfkill-any kbd-scrolllock kbd-numlock kbd-capslock
kbd-kanalock kbd-shiftlock kbd-altgrlock kbd-ctrllock kbd-altlock
kbd-shiftllock kbd-shiftrlock kbd-ctrlllock kbd-ctrlrlock usb-gadget
usb-host mmc0 timer oneshot disk-activity ide-disk mtd nand-disk
heartbeat backlight [gpio] cpu cpu0 default-on panic 
root@beaglebone:~# echo 45 > /sys/class/leds/beaglebone\:green\:usr3/gpio
      \end{raw}

   Now press the button a few times. The LED should be on when the button
isn't pressed and off when the buutton i pressed.

   How do we reverse this configuration and have it come on when the burron
is pressed.
      \begin{raw}
root@beaglebone:~# cat /sys/class/leds/beaglebone\:green\:usr3/inverted 
0
root@beaglebone:~# echo 1 > /sys/class/leds/beaglebone\:green\:usr3/inverted 
      \end{raw}

   \end{sol}
\end{exe}

\begin{exe} {Turn on USR3 LED with gpio sysfs interface}

   Now we'll turn on USR3 which is connected to the 24th gpio gpiochip1.

   So the gpio number is 32 + 24 or gpio56.

   This gpio isn't yet setup so we have a little more work to do.

   \begin{sol}
      \begin{raw}
root@beaglebone:~# echo 56 > /sys/class/gpio/export 
root@beaglebone:~# ls /sys/class/gpio/gpio56
active_low  device  direction  edge  label  power  subsystem  uevent  value
root@beaglebone:~# cat /sys/class/gpio/gpio56/active_low 
0
root@beaglebone:~# cat /sys/class/gpio/gpio56/direction 
out
root@beaglebone:~# cat /sys/class/gpio/gpio56/value 
0
root@beaglebone:~# echo 1 > /sys/class/gpio/gpio56/value 
root@beaglebone:~# cat /sys/class/gpio/gpio56/value 
1
root@beaglebone:~# echo 0 > /sys/class/gpio/gpio56/value 
      \end{raw}
   \end{sol}
\end{exe}

\begin{exe} {Light up the USR3 LED with libgpiod}
   Now use gpioset from libgpiod to light up the LED.

   \begin{sol}
      \begin{raw}
root@beaglebone:~# gpioset -m wait gpiochip1 24=1
# Press enter to terminate
      \end{raw}
   \end{sol}
\end{exe}

\end{Lab}

