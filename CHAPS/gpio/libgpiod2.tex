\begin{frame}
   {libgpiod - gpioinfo}
   \begin{itemize}
      \item The libgpiod library and set of tools is now the official way
	      of dealing with gpios.
      \begin{raw}
debian@beaglebone:~$ gpioinfo
gpiochip0 - 32 lines:
        line   0:  "MDIO_DATA"       unused   input  active-high
        line   1:   "MDIO_CLK"       unused   input  active-high
...
gpiochip1 - 32 lines:
        line   0:   "GPMC_AD0"       unused   input  active-high
        line   1:   "GPMC_AD1"       unused   input  active-high
...
gpiochip2 - 32 lines:
        line   0:  "GPMC_CSN3"      "P2_20"   input  active-high [used]
        line   1:   "GPMC_CLK"      "P2_17"   input  active-high [used]
...
gpiochip3 - 32 lines:
        line   0:  "GMII1_COL"       unused   input  active-high
        line   1:  "GMII1_CRS"       unused   input  active-high
...
      \end{raw}
   \end{itemize}
\end{frame}

\cprotect\note{


}

